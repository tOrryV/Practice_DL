%!TEX root = ../thesis.tex
% створюємо перелік умовних позначень, скорочень і термінів
\noindent КЕМ --- механізм інкапсуляції ключів (key encaplusation mechanism)\\
IND-CCA ---  нерозрізнення  за допомогою адаптивно вибраного шифрованого тексту (indistinguishability under adaptive chosen ciphertext attack)\\
CCA --- атака за допомогою вибраного шифротексту (chosen ciphertext cttack)\\
IND-CPA --- нерозрізнення за допомогою вибраного відкритого тексту (indistinguishability under chosen plaintext attack) \\
ML-KEM --- механізм інкапсуляції ключів на основі решітки (module-lattice-based key-encapsulation mechanism)\\
MLWE --- модульне навчання на помилках (module learning with errors)\\
PKE --- схема шифрування з відкритим ключем (public-key encryption)\\
LWE --- задача навчання на помилках (learning with errors)\\
FO --- перетворення Фудзісакі-Окамото (Fujisaki-Okamoto transform) \\