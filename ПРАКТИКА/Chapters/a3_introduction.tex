%!TEX root = ../thesis.tex
% створюємо вступ
Сучасні криптографічні протоколи встановлення ключів стикаються з викликами, зумовленими розвитком квантових обчислень. Традиційні асиметричні схеми, що забезпечують конфіденційність переданих даних, можуть стати вразливими в умовах появи достатньо потужних квантових комп'ютерів, здатних розв'язувати складні криптографічні задачі. У зв'язку з цим активно розвиваються гібридні схеми і протоколи встановлення ключів, що поєднують класичні та постквантові механізми задля досягнення довготривалої безпеки.

Одна із таких схем представлена у статті Joppe Bos, Leo Ducas та інших <<CRYSTALS – Kyber: a CCA-secure module-lattice-based KEM>> \cite{KyberCCA}. Вона пропонує гібридне поєднання класичних та постквантових компонентів для обміну ключами. Однак для практичного впровадження необхідно оцінити безпеку такої конструкції відносно стандартних моделей стійкості, зокрема IND-CCA та IND-CPA.

\textbf{Метою даної переддипломної практики} є реалізація спрощеної версії даної схеми Kyber та її формальна перевірка на відповідність вимогам IND-CCA та IND-CPA. Дослідження спрямоване на виявлення можливих вразливостей та підтвердження криптографічної стійкості отриманої конструкції.

\textbf{Об’єктом дослідження} є процеси оцінки криптографічної стійкості гібридних схем.

\textbf{Предметом дослідження} є методи аналізу стійкості до IND-CPA та IND-CCA атак.

\textbf{Реалізація алгоритму та експериментальні атаки} доступні у відкритому репозиторії за посиланням: 

\url{https://github.com/tOrryV/Practice_DL.git}.

