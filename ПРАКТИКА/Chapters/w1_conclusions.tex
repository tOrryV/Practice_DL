%!TEX root = ../thesis.tex
% створюємо Висновки до всієї роботи

У рамках переддипломної практики було реалізовано та досліджено спрощену версію постквантової криптосистеми Kyber — Baby Kyber. Було проведено теоретичний аналіз математичних основ алгоритму, визначено обмеження та особливості спрощеної моделі, зокрема фіксацію параметрів, відсутність перетворення Фур'є (NTT) та відмову від використання FO-перетворення.

Було розроблено програмну реалізацію алгоритму, яка охоплює генерацію ключів, шифрування та розшифрування повідомлень у вибраному параметричному просторі. Реалізація виконана мовою Python у середовищі PyCharm, із дотриманням модульного підходу до організації коду.

В межах експериментальної частини було проведено тестування стійкості алгоритму до атак типу IND-CPA та IND-CCA. За результатами аналізу встановлено, що реалізація демонструє статистичну стійкість до IND-CPA атак — ймовірність успішного вгадування не перевищує рівень випадковості. Водночас, алгоритм є повністю вразливим до IND-CCA атак через відсутність механізмів активного захисту, зокрема FO-перетворення.

Таким чином, реалізований алгоритм забезпечує базовий рівень конфіденційності у пасивній моделі супротивника, але не гарантує безпеки в умовах активного впливу. Робота демонструє практичне підтвердження важливості використання додаткових криптографічних трансформацій для досягнення повноцінної стійкості сучасних постквантових криптосистем.